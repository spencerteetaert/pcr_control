\section*{Abstract}
\addcontentsline{toc}{subsection}{\protect\numberline{}Abstract}

Continuum robots prove difficult in modelling and control because of their extremely non-linear behaviour and complex modelling requirements. Traditional beam modelling methods are either computationally inefficient or are incapable of accurately controlling non-simulated robots, even ones that are well constructed. The added complexity of trying to model inaccuracies in robot construction makes controlling continuum robots challenging. Parallel continuum robots add the opportunity of higher applied forces to the compliant robots, but they only add to the complexity of the control problem. With the rise of machine learning many deep neural networks have been demonstrated effective in modelling complex relationships. In this work we take steps towards learning-based control of a planar parallel tendon-driven continuum robot. This work introduces a robot prototype designed to be a standard research platform for learning based control design for PCRs, a dataset collected with the intent of use in both state estimation and control tasks, two baseline controllers to use as a reference point for future learning based work, and several negative learning-based control experiment results to inform future work. The contributions in this thesis lay the foundations for learning-based state estimation and control for a parallel continuum robot. All project code is available freely at: \url{https://github.com/spencerteetaert/pcr_control}.

\textbf{Keywords}: parallel continuum robots, differential control, machine learning, learning-based control, benchmarks