\subsection{Baseline Controllers}
\subsubsection{Overview}
The baseline controllers proposed in Section \ref{sec:baselines} do not have good tracking capabilities. For even the simplest tracked trajectory the controllers were only able to achieve an RMSE of \SI{0.043}{m}. This equates to a \SI{5.3}{\%} error at the end effector. Compared to the controllers presented in \cite{grassmann2022a, 7112506, 8115276} which achieved relative errors of \SI{0.4}{\%}, \SI{2.5}{\%}, and \SI{0.61}{\%} - \SI{0.76}{\%} respectively, this is worse. Comparisons made between these proposals though must be taken with a grain of salt as there exists vast differences between the robot's used in each work. The slightly lower performance on this system could be due to several reasons, including less time spent tuning controller gains or system errors that result in larger discrepancies. 

Both of the proposed baseline models achieved per iteration run times of faster than the \SI{1000}{Hz} run time constraint, with the PID controller achieving the fastest average runtime. The differential CC controller sees a slower end point update step because of its requirement to numerically solve for the robot's curvature at each motor feedback step. Even the learning-based controller, even though it did not work, had run times that satisfied the \SI{1000}{Hz} run time constraint. 

\subsubsection{Significance}
The baseline models proposed here can only be used for motion of the robot in closed-loop scenarios, an application space that is hardly seen in the real world. Figure \ref{fig:repetition_trial} demonstrates that even in the simple task of moving in a straight line, these controllers fall short. The static and dynamic effects of the system result in the addition of non-linear path errors when operating in task space. To counteract these errors, more complex modeling beyond these baseline's capabilities must be employed. As such, the proposed controllers do not contain much value beyond acting as a reference for future work and a motivating example for why learning-based solutions may be the optimal path forward for control of CRs. 

\subsubsection{Limitations}
\paragraph{High control error} leaves the baseline controllers little to be desired in most conventional control tasks. They struggle with small motions, unable to account for the spring force of the beam. More work could be done to improve the gains or runtime of the proposed controllers. This may result in marginal performance improvement, but they still fundamentally lack the ability to model the complex static and dynamic effects of this system. There is an upper bound to how well these baselines will be able to control the robot. 

\paragraph{No convergence guarantee} is given for the differential CC controller. This has not caused issues in practice on the machine used for this thesis, but is more likely if the controller is run on slower systems that reduce the solver runtime to ensure real time operation. Future work could explore theoretical guarantees of stability for each of these controllers. 

\paragraph{The Python implementation} of this entire control pipeline could have severe consequences for real time operation. In addition to having slower run times, Python does not have real-time capability. Python was the language of choice for this work primarily because of its fast development times compared to compiled languages such as C++. Lacking real-time support added additional runtime constraints and threading management to the code that would not be required if it were written in another language. For all of the items proposed in this thesis, Python was not a limiting factor. If future work looks to implement more complex algorithms or on slower systems, the language choice could become a large detractor of the project. 
