\section{Introduction}

Continuum Robots (CRs) are a current topic of research because of their inherent compliance, ability to be manufactured on sub-millimeter scales, and ability to have non-linear shape deformations \cite{4058827, survey}. This makes them ideal robots to be used in surgical and inspection applications \cite{7314984, DONG2017218}. Because of a CR’s compliance, it is limited in the force it can exert on the environment through a given end effector. Parallel Continuum Robots (PCRs) see multiple CRs attached at a common end effector, maintaining system compliance while enabling the robot to apply higher forces to the environment \cite{6906943, survey}. 

Forward and inverse kinematic modelling of a robot can be used in model-based control to move a robot with specified motions. Static modelling allows for the consideration of forces while dynamic modelling considers temporal effects on the system. Current models of CRs are unable to account for a number of complex factors in the system, such as internal robot friction, surface friction, and external loads, without significant computational overhead \cite{10.3389/frobt.2020.630245}. To achieve computation speeds that enable real-time control, often assumptions are made to simplify the robot’s static model to a simple kinematic one \cite{9143427} or to represent the robot's state with geometric approximations \cite{10.3389/frobt.2020.630245, 9143427, slilge_2020}. These simplifications greatly improve computation time while sacrificing model predictive accuracy. It is desirable to have the means to model these systems in real-time without making accuracy-sacrificing assumptions that seldom hold true in real-world applications.

Machine learning approaches have promised faster results and more accurate approximations of complex systems \cite{9199280}. Several works have demonstrated learning-based systems and their value to both the control and state estimation tasks \cite{s21062085, 8972568, 8643440, doi:10.1146/annurev-control-090419-075625, TOQUICA2021106682}. It is therefor a natural next step for PCRs to take advantage of the advantages of learning-based solution to help address the shortcomings of existing approaches. While some work has been done to use learning in CR control \cite{10.3389/frobt.2021.730330, grassmann2022a, 7112506}, nothing has been proposed for PCRs. 

\subsection{Contributions}
To achieve learning-based control for this system while ensuring future works have an accurate comparison point, several foundational steps must be taken. This thesis takes several of those steps to advance the field of CRs towards learning-based solutions. The contributions of this thesis are:
\begin{enumerate}
    \item Re-design and validation of a prototype for a two DOF planar parallel tendon-driven continuum robot. The robot is designed as an easy to implement research platform by using off-the-shelf components and 3D printed parts and being built for modularity and rapid reconfiguration  
    \item Establishes two real-time baseline controllers utilizing PID control and the constant curvature assumption that satisfy a \SI{1000}{Hz} update rate and have zero closed-loop asymptotic tracking error in task space  
    \item Establishes a data set for both control and state estimation tasks in addition to providing a learning-based pipeline for these tasks
    \item Presents several negative learning-based control results to inform future work 
\end{enumerate}

% The project was started with the redesign of an existing planar parallel tendon-driven continuum robot initially constructed at the Continuum Robotics Lab \cite{crl}. The robot was prototyped and manufacturing began but the model was never completed or tested. As the robot was partially completed at the start of this thesis, justification for some of the design choices of this particular robot is outside of the scope of this report. Completion and validation of the physical prototype is the first major objective of this thesis. 

% Establishing a baseline controller serves to act as both a comparison point to the learning-based controller and to enable reasonable control of the robot for data collection. A baseline model has been developed using techniques consistent with other works in the CR field. Evaluation of the effectiveness of this baseline controller must be completed for use as a reference for future approaches. 

% Future work will explore the development and evaluation of a learning-based controller for the robot prototype to compensate for unmodelled effects in the kinematic model of a planar parallel continuum robot \cite{slilge_2020}. Learning makes sense in this scenario due to its relatively low computation time to approximate complex models. Data will be collected on the physical system by generating a sample of control inputs and recording the end effector position and motor feedback over time. Trajectories will be restricted to enforce $C^4$ smoothness \cite{8772208} to be in line with standard robotics practice. Collection of end effector position data will follow the procedure in \cite{grassmann2022a} and collection of motor feedback as in \cite{Grimminger_2020}. 

\subsection{Document Structure}
In Section \ref{sec:background} a background of concepts is provided alongside a review of current state-of-the-art modelling methods. Section \ref{sec:methods} describes the robot prototype, baseline controller derivations, and dataset generation methodology. Section \ref{sec:results} will present the results of the four aspects of this project. Section \ref{sec:discussion} will explore the significance of the experiment results and highlight the limitations of this project. Lastly, Section \ref{sec:conclusion} will explore potential future work that could follow the proposed work. 